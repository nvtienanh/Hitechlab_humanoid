%; whizzy -pdf

% Document class
\documentclass[runningheads,a4paper,obeyspaces]{llncs}

% Packages
\usepackage{microtype}
\usepackage{amssymb}
\setcounter{tocdepth}{3}
\usepackage{graphicx}
\usepackage{pdfpages}
\usepackage{xspace}
\usepackage{url}
\usepackage{wrapfig}
\usepackage[colorlinks=true,linkcolor=black,citecolor=black,urlcolor=black]{hyperref}

% Commands
\newcommand{\keywords}[1]{
\par\addvspace\baselineskip\noindent\keywordname\enspace\ignorespaces#1}
\newcommand{\code}[1]{\protect\path{#1}}
\newcommand{\nop}{NimbRo-OP\xspace}
\newcommand{\SCL}{State Controller Library\xspace}

% Start of document
\begin{document}

% Main matter
\mainmatter

% Title information
\title{Soccer Behaviours Exercise Sheet}
\titlerunning{Soccer Behaviours Exercise Sheet}
\toctitle{Soccer Behaviours Exercise Sheet}

% Author(s)
\author{Philipp Allgeuer}
\authorrunning{P. Allgeuer}

% Affiliations
\institute{%
Autonomous Intelligent Systems, Computer Science, Univ.\ of Bonn, Germany\\
\url{pallgeuer@ais.uni-bonn.de} \\
\url{http://ais.uni-bonn.de}\\[3mm]
}

% Display the title
\maketitle

%%%%%%%%%%%%%%%%%%%%%%%%%%%%%%%%%%%%%%%%%%%%%%%%%%%%%%%%%%%%%%%%%%%%%%%%%%%%%%%%
%
\section{Introduction}

The goal of this exercise sheet is to introduce you to the programming of soccer
behaviours using \verb!C++! and the Robot Operating System (ROS) software
framework. We will be using the open source \SCL to create a finite state
machine that can control a \nop robot. 

The target outcome of this exercise, alongside an enriched understanding of the
\SCL and the \nop software framework, is the implementation of a simple ball
approach and kick behaviour that can avoid obstacles and kick the ball towards
the goal. A kinematic simulation of a \nop robot in a TeenSize soccer
environment will be used to evaluate your developed behaviour algorithm.

%%%%%%%%%%%%%%%%%%%%%%%%%%%%%%%%%%%%%%%%%%%%%%%%%%%%%%%%%%%%%%%%%%%%%%%%%%%%%%%%
%
\section{The State Controller Library}

The State Controller Library is a generic platform independent \verb!C++!
framework that allows for the realisation of finite state machines and multi-action planning
generalisations thereof. The library comes as part of the \nop
source distribution, but is also freely available as an independent \verb!C++!
library at \url{http://sourceforge.net/projects/statecontroller/}. It is
suggested at this point for you to look at the Doxygen documentation of the
\SCL, in particular the module description page, until you can answer the
following questions. This will make the remainder of the exercise easier to
complete. The documentation can be found in the source directory of the \nop
code under \code{src/nimbro/doc/out/html/group__StateControllerLibrary.html}, or
alternatively, if the documentation has not been compiled on your system yet,
you can access the documentation by downloading the library from the
aforementioned SourceForge page and opening the \code{doc/State Controller Library.html} file.\\

\noindent\textbf{Questions you should be able to answer:}
\begin{itemize}
\setlength{\itemsep}{5pt}
\item What is a state controller and what is it used for?
\item What is the required base class for every state controller object?
\item What is the recommended base class for every state controller state? What
arguments and template parameters do you need to supply to the constructor of
this class?
\item What is the state queue, and how do you transition to another state?
\item What are the two possible actions that can be returned from the
\code{execute()} callback of a state?
\item What are the state and the state controller callbacks, and when do each of
them get called?
\item How do you share data between states? How do you access state controller
data from a state?
\item What are the differences between state variables, state parameters and
shared variables?
\end{itemize}

\noindent Note however that as this application is fairly simple, it may not be
necessary to use more complex features of the library, such as for example
enqueueing multiple states into the state queue. The template source files for
this exercise (see later) may also help you with the process of familiarising
yourself with the library.

%%%%%%%%%%%%%%%%%%%%%%%%%%%%%%%%%%%%%%%%%%%%%%%%%%%%%%%%%%%%%%%%%%%%%%%%%%%%%%%%
%
\section{The ROS Framework}

Almost all of the calls to ROS-specific functions have been abstracted away from
the required exercise code, and can be found in \code{behaviour_exercise.cpp}.
The collection of functions and functionality that performs this abstraction
however must still be learnt. Before continuing you should familiarise yourself
with the following features.\\

\noindent\textbf{Useful additional features to understand:}
\begin{itemize}
\setlength{\itemsep}{5pt}
\item \textsf{Send gait command}: One of the first questions that naturally
arises is how to make the robot walk. This can be done using the
\code{sendGaitCommand()} function. It is recommended that you look into this
function's source code to understand what it does. Each of the gait target
velocity vector components are specified using dimensionless parameters on the
unit interval (i.e. from 0 to 1). A boolean flag is used to turn the gait on and
off as a whole.
\item \textsf{ROS console messages}: ROS provides support for displaying debug,
informative, warning and error messages in the console. Throttling and stream
support is also included for each variant of the console functions. For example,
the following warning will be printed at most every 0.5s, and uses the
\verb!C++! streaming syntax.
\begin{verbatim}
ROS_WARN_STREAM_THROTTLE(0.5, "Encountered bad x = " << x);
\end{verbatim}
Refer to \url{http://www.ros.org/wiki/rosconsole} for more information. 
\item \textsf{Plotter}: When ROS console messages are insufficient for
debugging, the plotter may be used. Refer to the \code{plotScalar()} and
\code{plotVector()} functions for more information on this. A time history of
the data passed to these functions appears in graphical form in the
visualisation GUI (see \emph{\nameref{sec:gettingstarted}}).
\item \textsf{Configuration server}: When writing a behaviour it is often useful
to be able to change certain parameters and try again without recompiling or
restarting the behaviour node. This allows for faster algorithm tuning and
design iterations. This can be done by defining configuration parameters, which
can subsequently be modified during runtime using the parameter tuner widget in
the visualisation GUI (see \emph{\nameref{sec:gettingstarted}}). Refer to the
definition and use of the \code{m_be_run()} parameter in the template exercise
code for an example of how to define a boolean configuration parameter.
Configuration parameters of floating point type can also be defined in an
analogous manner, except that the parameter class constructor needs to be called
with additional parameters specifying allowable ranges and so on.
\end{itemize}

%%%%%%%%%%%%%%%%%%%%%%%%%%%%%%%%%%%%%%%%%%%%%%%%%%%%%%%%%%%%%%%%%%%%%%%%%%%%%%%%
%
\section{Getting Started}\label{sec:gettingstarted}

For this section you will need an open console (e.g. Konsole) and an open code
editor (e.g. KDevelop). Note that at the end of your session you will have to
\textbf{save all files that you have modified onto a USB} to prevent losing them
when someone else uses the computer after you.

\subsection{Resetting the Workspace}
If someone has already modified the behaviours source code on your computer, or
if you just want to be sure, run the following:
\begin{verbatim}
$ nimbro source
$ git reset --hard HEAD
\end{verbatim}

\subsection{Running the Exercise Code}
\begin{enumerate}
\setlength{\itemsep}{5pt}
\item Navigate to the \code{behaviour_exercise} package folder in the
editor and open \code{behaviour_exercise_template.h} and
\code{behaviour_exercise_template.cpp}. Examine those source files and see what
you think they will do.
\begin{verbatim}
$ roscd behaviour_exercise
\end{verbatim}

\item Run the following command:
\begin{verbatim}
$ nimbro make
\end{verbatim}
This command should successfully build the \nop code, including the behaviours
exercise template code.

\item In four \textbf{separate} consoles (or console tabs), run the following
commands in the order given. Always wait for each process (or `node') to start
up before proceeding to the next. Note that tab completion may come in handy
here.
{\footnotesize
\begin{verbatim}
$ roscore
$ roslaunch behaviour_exercise behaviour_exercise_robot.launch
$ roslaunch behaviour_exercise behaviour_exercise_behaviour.launch
$ roslaunch behaviour_exercise behaviour_exercise_visualisation.launch
\end{verbatim}}
You should see that the robot process loads several motions and then finishes
initialising, the behaviour process starts running the idle state, and the
visualisation process opens up a graphical user interface (GUI) with a 3D
visualisation. Play around with the parameter tuner, plotter and RViz widgets,
in particular looking for configuration parameters and plotted variables with
the keywords `Behaviour Exercise'.

\item Press the \textsf{Fade In} button on the left hand side of the screen (in
the diagnostics widget) to make the robot stand up. It is only necessary to do
this once for every time that you restart the robot node. Once the robot is in
the standing state you may signal to the behaviour node that it should start
controlling the robot by checking the \textsf{run} configuration parameter. As
of yet the robot should not react though, as you have not written any control
code yet! By modifying some of the other configuration parameters however, you
may observe how the field is randomly initialised when \textsf{run} is checked.
\end{enumerate}

\subsection{Customising the Template Exercise Code}
You may have noticed that the behaviours node declares whose solution it is,
using a call to \textsf{ROS\_WARN}. Find the corresponding line in the source
code and add your name and date in place of the default text.

%%%%%%%%%%%%%%%%%%%%%%%%%%%%%%%%%%%%%%%%%%%%%%%%%%%%%%%%%%%%%%%%%%%%%%%%%%%%%%%%
%
\section{Writing the Ball Approach and Kick Behaviour}
\begin{enumerate}
\setlength{\itemsep}{5pt}
\item Following the hints given in the template code, create a new state called
\code{SearchForBallState}. Make sure you display a ROS info message at the start
of the \code{execute()} callback for debugging purposes. Also, make the robot
turn on the spot when the state controller is in this state. Now uncomment the
four commented lines in the \code{execute()} callback of the idle state in order
to allow the state controller to enter the search for ball state when
appropriate. Make and launch the code as described in the
\emph{\nameref{sec:gettingstarted}} section, and verify that the robot turns on
the spot when \textsf{run} is checked. You may choose to implement a more
advanced search for ball behaviour when you have completed a few more of the
following exercise questions, and have a functioning behaviour node.

\item Create another state called \code{ApproachBallState}, once again adding
code to display a (throttled) ROS info message to display the current state. For
now just make the robot walk straight ahead with constant velocity when it is in
the ball approach state. Add code to \code{SearchForBallState} to transition to
the ball approach state when the ball is seen, and add code to
\code{ApproachBallState} to transition back to the search for ball state when
the ball has not been seen for 1.0s. Refer to the \code{RobotSC::haveBall()}
function for this. Make and launch the code as described in the
\emph{\nameref{sec:gettingstarted}} section, and verify that the robot now turns
on the spot until it sees the ball, then starts walking straight ahead until it
cannot see the ball again, at which point it starts turning on the spot again,
and so on.

\item Write a more sophisticated ball approach behaviour that aims to position
the right (or left) foot of the robot behind the ball, with the robot facing the
nominated goal. The \code{ballVector()}, \code{goalVector()}, \code{haveBall()}
and \code{haveGoal()} functions will be of great use here. Make and launch the
code, and verify that your algorithm works with randomised robot and ball
positions, as well as when the \code{random_goal} configuration parameter set.

\item Add code to the ball approach state that transitions to a new state called
\code{KickBallState}, whenever the robot is positioned well enough behind the
ball to be able to kick it into the goal. In the \code{execute()} callback, use
the \code{robotIsStanding()} and \code{playRightKick()} functions to execute the
required kick. It may prove useful to override the \code{activate()} callback of
this state, for example so that you can send a gait command to stop walking only
exactly when the kick ball state is entered. Add code to return to the search
for ball state when the kick is over (i.e. the robot is standing again).
\end{enumerate}

%%%%%%%%%%%%%%%%%%%%%%%%%%%%%%%%%%%%%%%%%%%%%%%%%%%%%%%%%%%%%%%%%%%%%%%%%%%%%%%%
%
\section{Taking it Further}

If you have not done so already, now enable the
\code{use_imperfect_measurements} configuration parameter. This simulates noise
and uncertainty in the object detections. You can now also enable the simulation
of an obstacle using the \code{use_obstacle} parameter. You will need to improve
your ball approach behaviour to be able to avoid the obstacle, avoid kicking the
ball into the obstacle, and deal with the fact that it may be obstructing vision
of the ball. This would be an appropriate time also to improve your search for
ball behaviour. Further improvements could also include dynamically choosing
which foot to kick with, rather than always using a fixed foot.

%%%%%%%%%%%%%%%%%%%%%%%%%%%%%%%%%%%%%%%%%%%%%%%%%%%%%%%%%%%%%%%%%%%%%%%%%%%%%%%%
%
\section{If You Get Stuck}

If you get stuck, then you can always ask the supervisor of the exercise for
help. If you run out of time to finish the exercise or need an example to look
at, then you may refer to the \code{behaviour_exercise_sample.h} and
\code{behaviour_exercise_sample.cpp} source files, which contain a complete
sample solution to the exercise. You can compile the sample in place of your
exercise code by modifying \code{CMakeLists.txt} on the marked line, and
changing the header included in the \code{behaviour_exercise.cpp} source file.
Revert these changes to compile your exercise again. No cheating if you are
still working on the exercise though!\\

\noindent Enjoy!\\

\noindent 20/07/13

\end{document}
% EOF